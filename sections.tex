% sections.tex

\documentclass[centered,11pt,overlapped]{res}
%\documentclass[overlapped,line,11pt,draft]{res}
%-----------------------------------------------------------------------------
\usepackage[colorlinks,urlcolor=black,bookmarks=true]{hyperref}
\usepackage{enumitem}


%-----------------------------------------------------------------------------
%% % To enable extended pdf support, add pdftex and the following to the
%% % hyperref \usepackage line in the [..] section:
%%
\hypersetup{
pdftitle={Curriculum Vitae},
pdfauthor={Dong Zhou},
pdfcreator={pdfLaTeX},
%% pdfproducer={$Id: cv-us.tex,v 1.8 2004/12/31 17:07:33 jrblevin Exp $},
pdfsubject={Curriculum Vitae},
}
%% pdfkeywords={Applied Mathematics Economics Computer Science
%%     Differential Equations TANH method hyperbolic tangent centroid
%%     decomposition updating singular value decomposition linear algebra
%%     microeconomics System Administration Linux Unix German}
%-----------------------------------------------------------------------------
\def\Cplusplus{{\rm C\raise.5ex\hbox{\small ++}}}
%-----------------------------------------------------------------------------
\begin{document}

\name{{\Large \bf Dong Zhou}}

%\address{
%}
%\address{
%}
% {\verb+http://www.physics.wisc.edu/~nos+}

\begin{resume}

\begin{ncolumn}{2}
   {\tt \href{mailto:zhou.dong@gmail.com}{zhou.dong@gmail.com}} &  (917) 207-8391 (mobile) \\
   {\tt \href{https://nosarthur.github.io}{nosarthur.github.io}} &  US green card holder
\end{ncolumn}

%------------------------------------------------------------------------------

\section{\sc Summary}
A scientist and programmer.
Familiar with magnetic resonance imaging, quantum computing, and biophysics.

%------------------------------------------------------------------------------


\section{\sc Skills}
Python, C/\Cplusplus, Go, Matlab, Shell scripts, AWS, SQL\\
Computational physics/mathematics, Image processing, Machine learning\\
Mathematical modeling, Optimization with regularization, Stochastic processes

%---------------------------------------------------------------------------
%% %% Other possible sections:
%% \section{\sc Special Skills}
%% \section{\sc Journal Publications}
%% \section{\sc Refereed Conference Proceedings}
%% \section{\sc Conference Presentations}
%------------------------------------------------------------------------------

\section{\sc Experience}

\begin{itemize}[leftmargin=-0.02in]
\item Senior scientist, Schr\"odinger Inc. \hfill 2016-2018

Implement analysis toolkit for molecular dynamics trajectories using Python.
Maintain scientific computing services using AWS, Go, Python, PostgreSQL, and Polymer.js.
Maintain atom mapping module (subgraph isomorphism) for free energy perturbation.

\item Postdoc in radiology, Weill Medical College of Cornell University \hfill 2012--2016

Solved ill-posed inverse problems in medical imaging using Matlab, C/\Cplusplus.
Developed probes for transcranial magnetic stimulation, both in simulation using COMSOL multiphysics, and on hardware.

\item Postdoc in physics, Yale University \hfill 2011--2012

Simulated solid state quantum optics (circuit QED) systems using Python.
Developed schemes for quantum control and quantum gate.
\item {Research assistant, University of Wisconsin-Madison} \hfill {2007--2011} 

Studied quantum systems in the presence of classical noises (stochastic processes) using Matlab and \Cplusplus.
Developed scheme for quantum gate and quantum control.
Developed algorithm for graph isomorphism problem.
Performed X-ray diffraction and synchrotron radiation (X-ray absorption near edge spectroscopy and microscopy) experiments on nacre and other biological samples.
\end{itemize}

\ite{\bl{review}}{        
Reviewer for J. Phys. A: mathematical and theoretical, IEEE TBME, PLOS ONE, Quantum Information Processing, New J. Phys., Medicine, J. Neurosci. Methods, NeuroImage, Magnetic Resonance in Medicine, Medical Physics, NMR in Biomedicine, Int. J. Mod. Phys. B
}

\section{\sc Education}

\begin{itemize}[leftmargin=-0.02in]
\item {Ph.D in physics, University of Wisconsin-Madison (GPA 4.0)} \hfill {2006--2011}
\item {Graduate study in physics, University of Georgia-Athens (GPA 4.0)} \hfill {2004--2006}
\item {B.S. in physics, Honored Mixed Class}, {Zhejiang University, China (GPA 3.8)} \hfill {2000--2004}
\end{itemize}

%---------------------------------------------------------------------------

\section{\sc Honors and Awards}
\begin{itemize}[leftmargin=-0.02in]
\item International Society for Magnetic Resonance in Medicine (ISMRM) Merit Award, Magna Cum Laude, 2014
\item International Student Academic Achievement Award, UW-Madison, 2011
\item Ray and Anne Herb Award for Wisconsin Distinguished Graduate Fellowship, 2008
\item Emanuel R. Piore Award for Highest Scorer on the Qualifier Exam, UW-Madison, 2007
\item University Housing's Favorite Instructor Award for Fall 2006, UW-Madison, 2006
\item Van Vleck Fellowship for Graduate Students in Physics, UW-Madison, 2006
\item Honored Graduate of Zhejiang University, China, 2004
\item Honorary Enrollment, Zhejiang University, China, 2000
\item Tan Jiazhen (C. C. Tan) Scholarship for Outstanding High School Student in Biology, 1999
\item Kang Hui Scholarship for Highest Scorer in High School Entrance Exam, Hangzhou, China, 1996
\end{itemize}


%---------------------------------------------------------------------------
\section{\sc Patents}
\begin{enumerate}[leftmargin=-0.02in]
\item {\em Magnetic resonance imaging systems and methods for optimized parallel receive, excite, and shim (oPRES) } \\
Hui Han, Yi Wang, John Stager, Junghun Cho, and {\bf Dong Zhou}, pending
\end{enumerate}

%---------------------------------------------------------------------------
\ite{\bl{pubs}}{

\section{\sc Publications}
\begin{enumerate}[leftmargin=-0.02in]
\item{\em Cardiac Quantitative Susceptibility Mapping (QSM) for Heart Chamber Oxygenation}\\
Y. Wen, T.D. Nguyen, Z. Liu, P. Spincemaille, {\bf D. Zhou}, A. Dimov, Y. Kee, K. Deh, J. Kim, J. Weinsaft, and Y. Wang, Magn Reson Med 79 (3), 1545 (2018)
\item {\em Quantitative Susceptibility Mapping (QSM)-Based Cerebral Metabolic Rate of Oxygen Mapping with Minimum Local Variance }\\
J. Zhang, J. Cho, {\bf D. Zhou}, T.D. Nguyen, P. Spincemaille, A. Gupta, and Y. Wang, MRM 79 (1), 172 (2018)
\item {\em Susceptibility underestimation in a high susceptibility phantom: dependence on imaging resolution, magnitude contrast and other parameters}\\
{\bf D. Zhou}, J. Zhang, P. Spincemaille, Y. Wang, Magn Reson Med, 78 (3), 1080 (2017)
\item {\em Preconditioned Total Field Inversion (TFI) Method for Quantitative Susceptibility Mapping} \\
Z. Liu, Y. Kee, {\bf D. Zhou}, Y. Wang, and P. Spincemaille, Magn Reson Med 78 (1), 303 (2017)
\item {\em Cerebral Metabolic Rate of Oxygen (CMRO2) Mapping with Hyperventilation Challenge using Quantitative Susceptibility Mapping (QSM)}\\
J. Zhang, {\bf D. Zhou}, T.D. Nguyen, P. Spincemaille, A. Gupta, Y. Wang, Magn Reson Med, 77 (5), 1762 (2017)
\item {\em Three-dimensional MR Phase Unwrapping via Dual Decomposition} \\
J. Dong, F. Chen,  {\bf D. Zhou}, T. Liu, Z. Yu, and Y. Wang, Magn Reson Med 77 (3), 1353 (2017).
\item {\em On the influence of zero-padding on the nonlinear operations in Quantitative Susceptibility Mapping}\\
S. Eskreis-Winkler, {\bf D. Zhou}, T. Liu, A. Gupta, S. A. Gauthier, Y. Wang, and P. Spincemaille, MRI 35, 154 (2017)
\item {\em Quantitative susceptibility mapping and R2* measured changes during white matter lesion development in multiple sclerosis: myelin breaking down, myelin debris degradation and removal, and iron accumulation }\\
Y. Zhang, S.A. Gauthier, A. Gupta, W. Chen, J. Comunale, G.C.-Y. Chiang, {\bf D. Zhou}, G. Askin, W. Zhu, D. Pitt, Y. Wang, AJNR 37 (9) 1629 (2016).
\item {\em Longitudinal change in magnetic susceptibility of new enhanced multiple sclerosis (MS) lesions measured on serial quantitative susceptibility mapping (QSM)}\\
Y. Zhang, S.A. Gauthier, A. Gupta, J. Comunale, G. C.-Y. Chiang, {\bf D. Zhou}, W. Chen, A.E. Giambrone, W. Zhu, Y. Wang, JMRI 44 (2) 426 (2016).
\item {\em Increase in magnetic susceptibility after MS lesion formation and potential diagnostic utility} \\
Y. Zhang, S. Gauthier, L. Tu, A. Gupta, J. Comunale, G.C.-Y. Chiang, {\bf D. Zhou}, Y. Wang, MULTIPLE SCLEROSIS JOURNAL 21 502 (2016).
\item {\em Simultaneous Phase Unwrapping and Removal of chemical Shift (SPURS) using Graph Cuts: Application in Quantitative Susceptibility Mapping}\\
J. Dong, T. Liu, F. Chen, {\bf D. Zhou}, A. Dimov, A. Raj, Q. Cheng, P. Spincemaille, and Y. Wang, IEEE TMI 34 (2) 531 (2015).
\item {\em Background field removal by solving the Laplacian boundary value problem}\\
{\bf D. Zhou}, T. Liu, P. Spincemaille, and Y. Wang, NMR in Biomedicine, 27 (3), 312 (2014).
\item {\em An Iterative Spherical Mean Value (iSMV) Method for Background Field Removal in MRI}\\
Y. Wen, {\bf D. Zhou}, T. Liu, P. Spincemaille, and Y. Wang, Magn Reson Med 72 (4) 1065 (2014). 
\item {\em Magnetic susceptibility anisotropy: cylindrical symmetry from macroscopically ordered anisotropic molecules and accuracy of MRI measurements using few orientations}\\
C. Wisnieff, T. Liu, P. Spincemaille, S. Wang, {\bf D. Zhou}, and Y. Wang,  NeuroImage 70, 363 (2013).
\item{\em Mediated gates between spin qubits} \\
J. Fei, {\bf D. Zhou}, Y.-P. Shim, S. Oh, X. Hu, and M. Friesen, Phys. Rev. A 86, 062328 (2012). arXiv:1207.6063 
\item{\em Cavity-assisted quantum bath engineering with a superconducting qubit}\\
K. W. Murch, U. Vool, {\bf D. Zhou}, S. J. Weber, S.M. Girvin, and I. Siddiqi,
Phys. Rev. Lett. 109, 163602 (2012); arXiv:1207.0053
\item{\em Phenomenological noise model for superconducting qubits:
two-state fluctuators and 1/f noise}\\
{\bf D. Zhou} and R. Joynt, Supercond. Sci. Techno. 25, 045003 (2012); arXiv:1102.5766 
\item{\em Topology of entanglement evolution of two qubits}\\
{\bf D. Zhou}, G.-W. Chern, J. Fei, and R. Joynt, 
        Int. J. Mod. Phys. B 26, 1250054 (2012); arXiv:1007.1749
\item{\em Disappearance of entanglement: a topological point of view} \\
{\bf D. Zhou} and R. Joynt, QIP 11, 571 (2012); arXiv:1006.5474
\item{\em Suppression of decoherence and disentanglement by the exchange interaction}\\
A. De, A. Lang, {\bf D. Zhou}, and R. Joynt, Phys. Rev. A 83, 042331 (2011); arXiv:1006.5943
\item {\em Quasi-Hamiltonian Method for Computation of Decoherence Rates.}\\
R. Joynt, {\bf D. Zhou} and Q.-H. Wang, Int. J. Mod. Phys. B 25, 2115 (2011); arXiv:0906.2843
\item {\em Noise-induced looping on the Bloch sphere: Oscillatory effects in dephasing of qubits subject to broad-spectrum noise.}\\
{\bf D. Zhou} and R. Joynt, Phys. Rev. A 81, 010103 (2010); arXiv:0907.0463 
\item {\em Nacre Protein Fragment Templates Lamellar Aragonite Growth}\\
RA Metzler, JS Evans, CE Killian, {\bf D Zhou}, TH Churchill, N Appathurai, SN Coppersmith, PUPA Gilbert, J. Am. Chem. Soc. 132, 6329-6334 (2010). 
\item {\em X-ray photoelectron emission spectromicroscopic analysis of arborescent lycopsid cell wall composition and Carboniferous coal ball preservation.}\\
C. K. Boyce, M. Abrecht, {\bf D. Zhou}, and P.U.P.A. Gilbert, Int. J. Coal Geol. 83, 146-153 (2010).
\item {\em Disentanglement and decoherence from classical non-Markovian noise: Random telegraph noise.}\\
{\bf D. Zhou}, A. Lang, and R. Joynt, QIP 9, 727 (2010); arXiv:0912.3313 
\item {\em Two-particle quantum walks applied to the graph isomorphism problem.} \\
J. Gamble, M. Friesen, {\bf D. Zhou}, R. Joynt, and S.N. Coppersmith, Phys. Rev. A 81, 052313 (2010); arXiv:1002.3003
\item {\em A high-resolution chemical and structural study of framboidal pyrite formed within a low-temperature bacterial biofilm.}\\
L.C.W. MacLean, T. Tyliszczak, P.U.P.A. Gilbert, {\bf D. Zhou}, T.J. Pray, T.C. Onstott, G. Southam.  Geobiology 6, 471-480 (2008).
\item {\em Gradual Ordering in Red Abalone Nacre.}\\
P.U.P.A. Gilbert, R. A. Metzler, {\bf D. Zhou}, A. Scholl, A. Doran, A. Young, M. Kunz, N. Tamura, S. N. Coppersmith. J. Am. Chem. Soc. 130, 17519-17527 (2008); arXiv:0710.4573
\item {\em Assignment of polarization-dependent peaks in carbon K-edge spectra from biogenic and geologic aragonite.}\\
{\bf D. Zhou}, R.A. Metzler, T. Tyliszczak, J. Guo, M. Abrecht, S.N. Coppersmith, P.U.P.A. Gilbert. J. Phys. Chem. B 112, 13128-13135 (2008). 
\item {\em Probing the organic-mineral interface in model biominerals.}\\
R. A. Metzler, I.-W. Kim, K. Delak, J.S. Evans, {\bf D. Zhou}, E. Beniash, F. Wilt, M. Abrecht, J.-W. Chiou, J. Guo, S.N. Coppersmith, P.U.P.A. Gilbert.  Langmuir 24, 2680-2687 (2008). 
\item {\em Polarization-dependent imaging contrast in abalone shells.}\\
R.A. Metzler, {\bf D. Zhou}, M. Abrecht, J.-W. Chiou, J. Guo, D. Ariosa, S.N. Coppersmith, P.U.P.A. Gilbert. Phys. Rev. B 77, 064110 (2008).
\end{enumerate}

}

%---------------------------------------------------------------------------
\ite{\bl{short_pub_review}}{        

\section{\sc Publications and Services}
Author for 30+ journal articles and reviewer for 10+ journals.
}

%------------------------------------------------------------------------------

%==============================================================================
%\newpage
%\opening
%---------------------------------------------------------------------------

%%\vspace{\fill}\ \newline
%%{\tiny \rm $ $RCSfile: cv-us.tex,v $ $ }
%%{\tiny \rm $ $Date: 2004/12/31 17:07:33 $ $ }
%%{\tiny \rm $ $Revision: 1.8 $ $ }

%------------------------------------------------------------------------------

\ite{\bl{ref}}{        

%\newpage
%\opening
\section{\sc References}

%% Sample reference sheet:
%% http://www.uwm.edu/Dept/CDC/resrefer.htm 
%% References DO NOT belong on the resume. 
%% They should be listed on a separate sheet. 
%% Send the references with the resume only when 
%% specifically requested by the employer. 
%% There is no need to state "References available upon request." 
%% It is assumed that you will provide them. 


%
Reference A. Person \newline
Sandia National Laboratories \newline
000-000-0000 \newline
{\tt devnull@Sandia.GOV} \newline
Impressive Title for Person


%
Reference A. Person \newline
Sandia National Laboratories \newline
000-000-0000 \newline
{\tt devnull@Sandia.GOV} \newline
Impressive Title for Person

}

\end{resume}

\end{document}
